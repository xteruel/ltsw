% This is samplepaper.tex, a sample chapter demonstrating the
% LLNCS macro package for Springer Computer Science proceedings;
% Version 2.20 of 2017/10/04
\documentclass[runningheads]{etc/llncs/llncs}

% Used for displaying a sample figure. If possible, figure files should
% be included in EPS format.
\usepackage{graphicx}

% If you use the hyperref package, please uncomment the following line
% to display URLs in blue roman font according to Springer's eBook style:
%\renewcommand\UrlFont{\color{blue}\rmfamily}

\begin{document}
\title{Contribution Title\thanks{Supported by organization x.}}
% If the paper title is too long for the running head, you can set
% an abbreviated paper title here
%\titlerunning{Abbreviated paper title}

\author{
  First Author\inst{1}
  %\orcidID{0000-1111-2222-3333}
\and
  Second Author\inst{1}
  %\orcidID{1111-2222-3333-4444}
\and
  Third Author\inst{1}
  %\orcidID{2222--3333-4444-5555}
}
% First names are abbreviated in the running head.
% If there are more than two authors, 'et al.' is used.
\authorrunning{F. Author et al.}

\institute{
  Barcelona Supercomputing Center
  \email{e-mail@bsc.es} 
%\and
%  Springer Heidelberg, Tiergartenstr. 17, 69121 Heidelberg, Germany
%  \email{lncs@springer.com} \\
%  \url{http://www.springer.com/gp/computer-science/lncs}
%\and
%  ABC Institute, Rupert-Karls-University Heidelberg, Heidelberg, Germany
%  \email{\{abc,lncs\}@uni-heidelberg.de}
}

% %%%%%%%%%%%%%%%%%%%%%%%%%%%%%%%%%%%%%%%%%%%%%%%%%%%%%%%%%%%
\maketitle % ---- typeset the header of the contribution ----
% %%%%%%%%%%%%%%%%%%%%%%%%%%%%%%%%%%%%%%%%%%%%%%%%%%%%%%%%%%%
\nocite{*} % (XXX) To remove once bibliography is in place
\begin{abstract}
The abstract should briefly summarize the contents of the paper in 150--250 words.
\end{abstract}


\section{Introduction}

\section{Related Work}

\section{Discussion}

\section{Experimental Results}

\section{Conclusions and Future Work}

\section*{Acknowledgements}


% %%%%%%%%%%%%%%%%%%%%%%%%%%%%%%%%%%%%%%%%%%%%%%%%%%%%%%%%%%%
% ---- Bibliography ----
% %%%%%%%%%%%%%%%%%%%%%%%%%%%%%%%%%%%%%%%%%%%%%%%%%%%%%%%%%%%
% BibTeX users should specify bibliography style 'splncs04'.
% References will then be sorted and formatted in the correct
% style.
\bibliographystyle{etc/llncs/splncs04}
\bibliography{bib/default}
\end{document}
